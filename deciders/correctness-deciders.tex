% !TeX root = correctness-deciders.tex

\title{Correctness of bbchallenge's deciders}
\author{
        bbchallenge's contributors
}

\documentclass[a4paper,british]{article}
\usepackage{babel}
\usepackage[utf8]{inputenc}
\usepackage[margin=1in]{geometry}
%\usepackage{subfig}
\usepackage[hidelinks]{hyperref}
\usepackage{caption}
\usepackage{floatpag}
\usepackage{subcaption}
\usepackage{tikz}

\usepackage{algorithm}
\usepackage[noend]{algpseudocode}

\usepackage{graphicx}
\usepackage{mathtools}

\usepackage{amsmath,amsfonts,amssymb,amsthm}

\theoremstyle{definition} % don't use italics
\newtheorem{theorem}{Theorem}
\newtheorem{definition}[theorem]{Definition}
\newtheorem{lemma}[theorem]{Lemma}
\newtheorem{proposition}[theorem]{Proposition}
\newtheorem{corollary}[theorem]{Corollary}

\theoremstyle{remark} % emphasize the "Remark N." with italics, not bold
\newtheorem{observation}[theorem]{Observation}
\newtheorem{example}[theorem]{Example}
\newtheorem{remark}[theorem]{Remark}


\usepackage{microtype,xspace,wrapfig,multicol} 
\usepackage[textsize=tiny,color=lightgray]{todonotes} 
\usepackage[normalem]{ulem} % sout
\usepackage{stmaryrd}

\newcommand{\ts}[1]{{\color{red}#1}}
\newcommand{\tsi}[1]{\todo[inline]{TS: #1}}
\newcommand{\tsm}[1]{\todo{TS: #1}}
\newcommand{\tss}[2]{{\ts{\sout{#1}}} {\ts{#2}}}
\newcommand{\jb}[1]{{\color{blue}#1}}
\newcommand{\jbi}[1]{\todo[inline]{JB: #1}}
\newcommand{\jbm}[1]{\todo{JB: #1}}
\newcommand{\jbs}[2]{{\jb{\sout{#1}}} {\jb{#2}}}
\newcommand{\tabi}{\hspace{\algorithmicindent}}

\newcommand{\N}{\mathbb{N}}

\usepackage{xcolor}

\definecolor{colorA}{RGB}{255,0,0}
\definecolor{colorB}{RGB}{255,128,0}
\definecolor{colorC}{RGB}{0,0,255}
\definecolor{colorD}{RGB}{0,255,0}
\definecolor{colorE}{RGB}{255,0,255}

\begin{document}
\date{}
\maketitle

\begin{abstract}
  The Busy Beaver Challenge (or bbchallenge) aims at collaboratively solving the following conjecture: ``BB(5) = 47,176,870'' [Aaronson, 2020]\nocite{BusyBeaverFrontier}. This conjecture says that if a 5-state Turing machine runs for more than 47,176,870 steps without halting then it will never halt (starting from all-0 memory tape). Proving this conjecture amounts to decide whether or not 88,664,064 Turing machines with 5-state halt or not -- starting from all-0 tape. In order to decide the behavior of these machines we write \textit{deciders}. A decider is a program that takes as input a Turing machine and outputs \texttt{true} if it is able to tell whether the machine halts or not. Each decider is specialised in recognising a particular type of behavior that can be decided.

  In this document we are concerned with proving the correctness of these deciders programs. More context and information about this methodology are available at \url{https://bbchallenge.org}.
\end{abstract}
\tableofcontents

\section{Conventions}\label{sec:conventions}

\begin{table}[h!]
  \centering
  \begin{tabular}{lll}
      & 0     & 1   \\
    A & 1RB   & 1LC \\
    B & 1RC   & 1RB \\
    C & 1RD   & 0LE \\
    D & 1LA   & 1LD \\
    E & - - - & 0LA
  \end{tabular}
  \caption{Transition table of the current 5-state busy beaver champion: it halts after 47,176,870 steps.\\\url{https://bbchallenge.org/1RB1LC1RC1RB1RD0LE1LA1LD---0LA&status=halt}}
\end{table}\label{table:bb5}

The set $\mathbb{N}$ denotes $\{0,1,2\dots\}$.

\paragraph*{Turing machines.}The Turing machines that are studied in the context of bbchallenge use a binary alphabet and a single bi-infinite tape. Machine transitions are either undefined (in which case the machine halts) or given by (a) a symbol to write (b) a direction to move (right or left) and (c) a state to go to. Table~\ref{table:bb5} gives the transition table of the current 5-state busy beaver champion. The machine halts after 47,176,870 steps (starting from all-0 tape) when it reads a 0 in state E, which is undefined.

A \textit{configuration} of a Turing machine is defined by the 3-tuple: (i) state (ii) position of the head (iii) content of the memory tape. In the context of bbchallenge, \textit{the initial configuration} of a machine is always (i) state is 0, i.e. the first state to appear in the machine's description (ii) head's position is 0 (iii) the initial tape is all-0 -- i.e. each memory cell is containing 0. We write $c_1 \vdash_\mathcal{M} c_2$ if a configuration $c_2$ is obtained from $c_1$ in one computation step of machine $\mathcal{M}$. We omit $\mathcal{M}$ if it is clear from context. We let $c_1 \vdash^s c_2$ denote a sequence of $s$ computation steps, and let  $c_1 \vdash^* c_2$ denote zero or more computation steps. % exact same wording as in https://dna.hamilton.ie/assets/dw/NearyWoodsFCT09.pdf
We write $c_1 \vdash \bot$ if the machine halts after executing one computation step from configuration $c_1$. In the context of bbchallenge, halting happens when an undefined machine transition  is met i.e. no instruction is given for when the machine is in the state, tape position and tape corresponding to configuration $c_1$.

\paragraph*{Space-time diagram.} We use space-time diagrams to give a visual representation of the behavior of a given machine. The space-time diagram of machine $\mathcal{M}$ is an image where the $i^\text{th}$ row of the image gives:
\begin{enumerate}
  \item The content of the tape after $i$ steps (black is 0 and white is 1).
  \item The position of the head is colored to give state information using the following colours for 5-state machines: \textcolor{colorA}{A},  \textcolor{colorB}{B},  \textcolor{colorC}{C},  \textcolor{colorD}{D},  \textcolor{colorE}{E}.
\end{enumerate}

\input{sections/decider-cyclers.tex}
\input{sections/decider-translated-cyclers.tex}
\input{sections/decider-backward-reasoning.tex}
\input{sections/decider-halting-segment.tex}
\input{sections/decider-FAR-finite-automata-reduction.tex}
% !TeX root = ../correctness-deciders.tex

\section{Bouncers}\label{sec:bouncers}

Intuitively, a \emph{bouncer} is a Turing machine that builds a tape with
ever-expanding repeating fragments. We shall refine this definition later,
such that it clearly explains what machines the decider should hope to decide.

\paragraph{Notation.} For a tape segment $s \in \{0, 1\}^*$, we will
write $(s)$ to mean $s$ repeated any number of times, i.e.~$s^k$ for
an unspecified $k$. 

\begin{example}
    $0\,(110)\,1$ can refer to $01$, $01101$, $01101101$, and so on.
\end{example}

\begin{definition}
    A bouncer's \emph{reset configuration} is given by
    a list of \emph{walls} $w_0,\ \ldots,\ w_n \in \{0, 1\}^*$
    and \emph{repeaters} $r_1,\ \ldots,\ r_n$, as well as a fixed state $q$,
    and refers to the configuration
    \begin{equation}
        0^\infty\; [0]\; w_0\; (r_1)\; w_1\; (r_2)\; w_2\: \cdots\: (r_n)\; w_n,
    \end{equation}
    in state $q$.
\end{definition}

% TODO: We would probably want an illustration here. Perhaps a space-time
% diagram with the reset configurations marked?


\bibliographystyle{abbrv}
\bibliography{correctness-deciders}

\end{document}
